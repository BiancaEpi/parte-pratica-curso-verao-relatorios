% Options for packages loaded elsewhere
\PassOptionsToPackage{unicode}{hyperref}
\PassOptionsToPackage{hyphens}{url}
%
\documentclass[
]{article}
\title{Relatório do curso}
\author{Bianca Bittencourt}
\date{2/2/2022}

\usepackage{amsmath,amssymb}
\usepackage{lmodern}
\usepackage{iftex}
\ifPDFTeX
  \usepackage[T1]{fontenc}
  \usepackage[utf8]{inputenc}
  \usepackage{textcomp} % provide euro and other symbols
\else % if luatex or xetex
  \usepackage{unicode-math}
  \defaultfontfeatures{Scale=MatchLowercase}
  \defaultfontfeatures[\rmfamily]{Ligatures=TeX,Scale=1}
\fi
% Use upquote if available, for straight quotes in verbatim environments
\IfFileExists{upquote.sty}{\usepackage{upquote}}{}
\IfFileExists{microtype.sty}{% use microtype if available
  \usepackage[]{microtype}
  \UseMicrotypeSet[protrusion]{basicmath} % disable protrusion for tt fonts
}{}
\makeatletter
\@ifundefined{KOMAClassName}{% if non-KOMA class
  \IfFileExists{parskip.sty}{%
    \usepackage{parskip}
  }{% else
    \setlength{\parindent}{0pt}
    \setlength{\parskip}{6pt plus 2pt minus 1pt}}
}{% if KOMA class
  \KOMAoptions{parskip=half}}
\makeatother
\usepackage{xcolor}
\IfFileExists{xurl.sty}{\usepackage{xurl}}{} % add URL line breaks if available
\IfFileExists{bookmark.sty}{\usepackage{bookmark}}{\usepackage{hyperref}}
\hypersetup{
  pdftitle={Relatório do curso},
  pdfauthor={Bianca Bittencourt},
  hidelinks,
  pdfcreator={LaTeX via pandoc}}
\urlstyle{same} % disable monospaced font for URLs
\usepackage[margin=1in]{geometry}
\usepackage{color}
\usepackage{fancyvrb}
\newcommand{\VerbBar}{|}
\newcommand{\VERB}{\Verb[commandchars=\\\{\}]}
\DefineVerbatimEnvironment{Highlighting}{Verbatim}{commandchars=\\\{\}}
% Add ',fontsize=\small' for more characters per line
\usepackage{framed}
\definecolor{shadecolor}{RGB}{248,248,248}
\newenvironment{Shaded}{\begin{snugshade}}{\end{snugshade}}
\newcommand{\AlertTok}[1]{\textcolor[rgb]{0.94,0.16,0.16}{#1}}
\newcommand{\AnnotationTok}[1]{\textcolor[rgb]{0.56,0.35,0.01}{\textbf{\textit{#1}}}}
\newcommand{\AttributeTok}[1]{\textcolor[rgb]{0.77,0.63,0.00}{#1}}
\newcommand{\BaseNTok}[1]{\textcolor[rgb]{0.00,0.00,0.81}{#1}}
\newcommand{\BuiltInTok}[1]{#1}
\newcommand{\CharTok}[1]{\textcolor[rgb]{0.31,0.60,0.02}{#1}}
\newcommand{\CommentTok}[1]{\textcolor[rgb]{0.56,0.35,0.01}{\textit{#1}}}
\newcommand{\CommentVarTok}[1]{\textcolor[rgb]{0.56,0.35,0.01}{\textbf{\textit{#1}}}}
\newcommand{\ConstantTok}[1]{\textcolor[rgb]{0.00,0.00,0.00}{#1}}
\newcommand{\ControlFlowTok}[1]{\textcolor[rgb]{0.13,0.29,0.53}{\textbf{#1}}}
\newcommand{\DataTypeTok}[1]{\textcolor[rgb]{0.13,0.29,0.53}{#1}}
\newcommand{\DecValTok}[1]{\textcolor[rgb]{0.00,0.00,0.81}{#1}}
\newcommand{\DocumentationTok}[1]{\textcolor[rgb]{0.56,0.35,0.01}{\textbf{\textit{#1}}}}
\newcommand{\ErrorTok}[1]{\textcolor[rgb]{0.64,0.00,0.00}{\textbf{#1}}}
\newcommand{\ExtensionTok}[1]{#1}
\newcommand{\FloatTok}[1]{\textcolor[rgb]{0.00,0.00,0.81}{#1}}
\newcommand{\FunctionTok}[1]{\textcolor[rgb]{0.00,0.00,0.00}{#1}}
\newcommand{\ImportTok}[1]{#1}
\newcommand{\InformationTok}[1]{\textcolor[rgb]{0.56,0.35,0.01}{\textbf{\textit{#1}}}}
\newcommand{\KeywordTok}[1]{\textcolor[rgb]{0.13,0.29,0.53}{\textbf{#1}}}
\newcommand{\NormalTok}[1]{#1}
\newcommand{\OperatorTok}[1]{\textcolor[rgb]{0.81,0.36,0.00}{\textbf{#1}}}
\newcommand{\OtherTok}[1]{\textcolor[rgb]{0.56,0.35,0.01}{#1}}
\newcommand{\PreprocessorTok}[1]{\textcolor[rgb]{0.56,0.35,0.01}{\textit{#1}}}
\newcommand{\RegionMarkerTok}[1]{#1}
\newcommand{\SpecialCharTok}[1]{\textcolor[rgb]{0.00,0.00,0.00}{#1}}
\newcommand{\SpecialStringTok}[1]{\textcolor[rgb]{0.31,0.60,0.02}{#1}}
\newcommand{\StringTok}[1]{\textcolor[rgb]{0.31,0.60,0.02}{#1}}
\newcommand{\VariableTok}[1]{\textcolor[rgb]{0.00,0.00,0.00}{#1}}
\newcommand{\VerbatimStringTok}[1]{\textcolor[rgb]{0.31,0.60,0.02}{#1}}
\newcommand{\WarningTok}[1]{\textcolor[rgb]{0.56,0.35,0.01}{\textbf{\textit{#1}}}}
\usepackage{longtable,booktabs,array}
\usepackage{calc} % for calculating minipage widths
% Correct order of tables after \paragraph or \subparagraph
\usepackage{etoolbox}
\makeatletter
\patchcmd\longtable{\par}{\if@noskipsec\mbox{}\fi\par}{}{}
\makeatother
% Allow footnotes in longtable head/foot
\IfFileExists{footnotehyper.sty}{\usepackage{footnotehyper}}{\usepackage{footnote}}
\makesavenoteenv{longtable}
\usepackage{graphicx}
\makeatletter
\def\maxwidth{\ifdim\Gin@nat@width>\linewidth\linewidth\else\Gin@nat@width\fi}
\def\maxheight{\ifdim\Gin@nat@height>\textheight\textheight\else\Gin@nat@height\fi}
\makeatother
% Scale images if necessary, so that they will not overflow the page
% margins by default, and it is still possible to overwrite the defaults
% using explicit options in \includegraphics[width, height, ...]{}
\setkeys{Gin}{width=\maxwidth,height=\maxheight,keepaspectratio}
% Set default figure placement to htbp
\makeatletter
\def\fps@figure{htbp}
\makeatother
\setlength{\emergencystretch}{3em} % prevent overfull lines
\providecommand{\tightlist}{%
  \setlength{\itemsep}{0pt}\setlength{\parskip}{0pt}}
\setcounter{secnumdepth}{-\maxdimen} % remove section numbering
\ifLuaTeX
  \usepackage{selnolig}  % disable illegal ligatures
\fi

\begin{document}
\maketitle

\hypertarget{primeiros-passos-no-rmarkdown}{%
\subsection{Primeiros passos no
Rmarkdown}\label{primeiros-passos-no-rmarkdown}}

\hypertarget{tuxedtulos-e-subtuxedtulos}{%
\subsubsection{Títulos e subtítulos}\label{tuxedtulos-e-subtuxedtulos}}

O número de \# definem os títulos e subtítulos. Dois \#\# definem os
subtitulos, três \#\#\# definem novos subtitulos e assim por diante.

\hypertarget{listas-numeradas}{%
\subsubsection{Listas numeradas}\label{listas-numeradas}}

É possivel fazer listas numerdas incluindo um número e ponto. Ex. 1. 1.
1. Precisa dar um enter no parágrafo e um tab.

\begin{enumerate}
\def\labelenumi{\arabic{enumi}.}
\tightlist
\item
  primeiro intem
\item
  segundo intem
\item
  terceiro intem
\end{enumerate}

\hypertarget{listas-nuxe3o-numeradas}{%
\subsubsection{Listas não numeradas}\label{listas-nuxe3o-numeradas}}

Também da pra fazer listas nao numeradas com um travessão:

\begin{itemize}
\tightlist
\item
  Item 1
\item
  Item 2
\item
  Item 3
\end{itemize}

\hypertarget{negrito}{%
\subsubsection{Negrito}\label{negrito}}

Para usar o \textbf{negrito} utiliza dois asteriscos.

\hypertarget{ituxe1lico}{%
\subsubsection{Itálico}\label{ituxe1lico}}

Para usar o \emph{itálico} utiliza um asterisco.

\hypertarget{destacar-comandos}{%
\subsubsection{Destacar comandos}\label{destacar-comandos}}

Utiliza-se o crases para indicar um comando a ser visualizado em
destaque no texto. Como \texttt{table(pinguins\$ilha)}

\hypertarget{citauxe7uxe3o}{%
\subsubsection{Citação}\label{citauxe7uxe3o}}

Para incluir citação inseri um \textgreater{}

\begin{quote}
Segundo o conteúdo do curso de verão USP
\end{quote}

\hypertarget{links}{%
\subsubsection{Links}\label{links}}

Para incluir links dois colchetes \emph{mostra o texto} e dois
parenteses \emph{inseri o link}.

\href{https://rmarkdown.rstudio.com/}{Acesso ao Rmarkdown}

\hypertarget{figuras}{%
\subsubsection{Figuras}\label{figuras}}

Para inserir figuras primeiro o simbolo de exclamação, seguido de
parenteses e colchetes.

{[}\includegraphics{https://camo.githubusercontent.com/de0519dd8e4ebc982eb0ddfaa9c6cd0924149e6c/68747470733a2f2f626f6f6b646f776e2e6f72672f79696875692f726d61726b646f776e2f696d616765732f6865782d726d61726b646f776e2e706e67}

\hypertarget{aula-1}{%
\subsubsection{Aula 1}\label{aula-1}}

\href{https://cursosextensao.usp.br/course/view.php?id=2991\&section=4}{Link
da primeira aula do curso}

\hypertarget{cuxf3digos-nos-chucks}{%
\section{Códigos nos chucks}\label{cuxf3digos-nos-chucks}}

Tem alguns comandos no chuck que evitam que mensagens indesejadas
apareçam. Ou pode definir se ocódigo deve ser executado ou se quer que o
comando fique visivel no documento.

\begin{Shaded}
\begin{Highlighting}[]
\FunctionTok{library}\NormalTok{(here)}
\end{Highlighting}
\end{Shaded}

\hypertarget{importando-a-base-de-dados}{%
\subsection{Importando a base de
dados}\label{importando-a-base-de-dados}}

Existe diferenca entre o diretório do projeto e o diretório do markdown.
O diretório do markdown é a pasta onde o arquivo foi salvo.

\begin{Shaded}
\begin{Highlighting}[]
\NormalTok{caminho }\OtherTok{\textless{}{-}}\NormalTok{ here}\SpecialCharTok{::}\FunctionTok{here}\NormalTok{(}\StringTok{"dados"}\NormalTok{, }\StringTok{"pinguins.csv"}\NormalTok{) }
\NormalTok{pinguins }\OtherTok{\textless{}{-}}\NormalTok{ readr}\SpecialCharTok{::}\FunctionTok{read\_csv}\NormalTok{(caminho)}
\end{Highlighting}
\end{Shaded}

\begin{verbatim}
## Rows: 344 Columns: 8
## -- Column specification --------------------------------------------------------
## Delimiter: ","
## chr (3): especie, ilha, sexo
## dbl (5): comprimento_bico, profundidade_bico, comprimento_nadadeira, massa_c...
## 
## i Use `spec()` to retrieve the full column specification for this data.
## i Specify the column types or set `show_col_types = FALSE` to quiet this message.
\end{verbatim}

\begin{Shaded}
\begin{Highlighting}[]
\NormalTok{dplyr}\SpecialCharTok{::}\FunctionTok{glimpse}\NormalTok{(pinguins)}
\end{Highlighting}
\end{Shaded}

\begin{verbatim}
## Rows: 344
## Columns: 8
## $ especie               <chr> "Pinguim-de-adélia", "Pinguim-de-adélia", "Pingu~
## $ ilha                  <chr> "Torgersen", "Torgersen", "Torgersen", "Torgerse~
## $ comprimento_bico      <dbl> 39.1, 39.5, 40.3, NA, 36.7, 39.3, 38.9, 39.2, 34~
## $ profundidade_bico     <dbl> 18.7, 17.4, 18.0, NA, 19.3, 20.6, 17.8, 19.6, 18~
## $ comprimento_nadadeira <dbl> 181, 186, 195, NA, 193, 190, 181, 195, 193, 190,~
## $ massa_corporal        <dbl> 3750, 3800, 3250, NA, 3450, 3650, 3625, 4675, 34~
## $ sexo                  <chr> "macho", "fêmea", "fêmea", NA, "fêmea", "macho",~
## $ ano                   <dbl> 2007, 2007, 2007, 2007, 2007, 2007, 2007, 2007, ~
\end{verbatim}

\hypertarget{cuxf3digos-no-meio-do-texto}{%
\subsubsection{Códigos no meio do
texto}\label{cuxf3digos-no-meio-do-texto}}

Aqui vou escrever um texto que apresenta um resultado no corpo do texto.

A base de dados pinguins apresenta dados referente à 344 pinguins, das
seguintes espécies: Pinguim-de-adélia, Pinguim-gentoo,
ePinguim-de-barbicha Os dados foram coletados entre os anos 2007e 2009.
O peso médio dos pinguins amostrados foi de 42 kg. Os dados foram
obtidos através do pacote Palmer Penguins.

\hypertarget{inserir-figuras-no-texto-com-ajustes-de-puxe1gina}{%
\section{Inserir figuras no texto com ajustes de
página}\label{inserir-figuras-no-texto-com-ajustes-de-puxe1gina}}

\begin{Shaded}
\begin{Highlighting}[]
\NormalTok{knitr}\SpecialCharTok{::}\FunctionTok{include\_graphics}\NormalTok{(}\StringTok{"https://d33wubrfki0l68.cloudfront.net/aee91187a9c6811a802ddc524c3271302893a149/a7003/images/bandthree2.png"}\NormalTok{)}
\end{Highlighting}
\end{Shaded}

\begin{figure}

{\centering \includegraphics[width=0.5\linewidth]{https://d33wubrfki0l68.cloudfront.net/aee91187a9c6811a802ddc524c3271302893a149/a7003/images/bandthree2} 

}

\caption{Ilustração dos pinguins}\label{fig:unnamed-chunk-1}
\end{figure}

\hypertarget{tabelas}{%
\section{Tabelas}\label{tabelas}}

\begin{Shaded}
\begin{Highlighting}[]
\NormalTok{torg }\OtherTok{\textless{}{-}}\NormalTok{ pinguins }\SpecialCharTok{|}\ErrorTok{\textgreater{}}
\NormalTok{  dplyr}\SpecialCharTok{::}\FunctionTok{filter}\NormalTok{(ilha }\SpecialCharTok{==} \StringTok{"Torgersen"}\NormalTok{) }\SpecialCharTok{|}\ErrorTok{\textgreater{}}
\NormalTok{  dplyr}\SpecialCharTok{::}\FunctionTok{select}\NormalTok{ (comprimento\_bico}\SpecialCharTok{:}\NormalTok{massa\_corporal)}

\NormalTok{knitr}\SpecialCharTok{::}\FunctionTok{kable}\NormalTok{(torg)}
\end{Highlighting}
\end{Shaded}

\begin{longtable}[]{@{}rrrr@{}}
\toprule
comprimento\_bico & profundidade\_bico & comprimento\_nadadeira &
massa\_corporal \\
\midrule
\endhead
39.1 & 18.7 & 181 & 3750 \\
39.5 & 17.4 & 186 & 3800 \\
40.3 & 18.0 & 195 & 3250 \\
NA & NA & NA & NA \\
36.7 & 19.3 & 193 & 3450 \\
39.3 & 20.6 & 190 & 3650 \\
38.9 & 17.8 & 181 & 3625 \\
39.2 & 19.6 & 195 & 4675 \\
34.1 & 18.1 & 193 & 3475 \\
42.0 & 20.2 & 190 & 4250 \\
37.8 & 17.1 & 186 & 3300 \\
37.8 & 17.3 & 180 & 3700 \\
41.1 & 17.6 & 182 & 3200 \\
38.6 & 21.2 & 191 & 3800 \\
34.6 & 21.1 & 198 & 4400 \\
36.6 & 17.8 & 185 & 3700 \\
38.7 & 19.0 & 195 & 3450 \\
42.5 & 20.7 & 197 & 4500 \\
34.4 & 18.4 & 184 & 3325 \\
46.0 & 21.5 & 194 & 4200 \\
35.9 & 16.6 & 190 & 3050 \\
41.8 & 19.4 & 198 & 4450 \\
33.5 & 19.0 & 190 & 3600 \\
39.7 & 18.4 & 190 & 3900 \\
39.6 & 17.2 & 196 & 3550 \\
45.8 & 18.9 & 197 & 4150 \\
35.5 & 17.5 & 190 & 3700 \\
42.8 & 18.5 & 195 & 4250 \\
40.9 & 16.8 & 191 & 3700 \\
37.2 & 19.4 & 184 & 3900 \\
36.2 & 16.1 & 187 & 3550 \\
42.1 & 19.1 & 195 & 4000 \\
34.6 & 17.2 & 189 & 3200 \\
42.9 & 17.6 & 196 & 4700 \\
36.7 & 18.8 & 187 & 3800 \\
35.1 & 19.4 & 193 & 4200 \\
38.6 & 17.0 & 188 & 2900 \\
37.3 & 20.5 & 199 & 3775 \\
35.7 & 17.0 & 189 & 3350 \\
41.1 & 18.6 & 189 & 3325 \\
36.2 & 17.2 & 187 & 3150 \\
37.7 & 19.8 & 198 & 3500 \\
40.2 & 17.0 & 176 & 3450 \\
41.4 & 18.5 & 202 & 3875 \\
35.2 & 15.9 & 186 & 3050 \\
40.6 & 19.0 & 199 & 4000 \\
38.8 & 17.6 & 191 & 3275 \\
41.5 & 18.3 & 195 & 4300 \\
39.0 & 17.1 & 191 & 3050 \\
44.1 & 18.0 & 210 & 4000 \\
38.5 & 17.9 & 190 & 3325 \\
43.1 & 19.2 & 197 & 3500 \\
\bottomrule
\end{longtable}

\end{document}
